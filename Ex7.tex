\begin{ex} \label{ex_7} \addcontentsline{mdf}{mdf}{Exercise \ref{ex_7}}
    Show that the lagrangian density
    \begin{equation}
    \label{lag_ex7}
        \La = \frac{1}{2} \left( \partial_\mu \phi_1 \partial^\mu \phi_1 + \partial_\mu \phi_2 \partial^\mu \phi_2 \right) - \frac{m^2}{2} \left( \phi_1^2 + \phi_2^2\right) - \lambda \left( \phi_1^2 + \phi_2^2\right)^2
    \end{equation}
    is invariant under
    \begin{equation}\label{rot}
        \begin{pmatrix} \phi_1' \\ \phi_2' \end{pmatrix} = R(\theta) \begin{pmatrix} \phi_1 \\ \phi_2 \end{pmatrix} \, ,
    \end{equation}
    where $R(\theta)$ is a rotation of an angle $\theta$. Determine the Noether current and charge.
\end{ex}

%%%%%%%%%%%%%%%%%%%%%%%%%%%%%%%%%%%%%%%%%%%%%%%%%%%%%%%%%%%%%%%%%%%%%%%%%%%%%%%%%%%%%%%%%%

\begin{sol}
    Consider the inverse of Eq.~\eqref{rot},\footnote{Also the repeated Latin indices are to be intended summed.}
    \begin{equation}
        \begin{pmatrix} \phi_1 \\ \phi_2 \end{pmatrix} = R^{-1}(\theta) \begin{pmatrix} \phi_1' \\ \phi_2' \end{pmatrix} \quad \implies \quad \phi_i = R^{-1}(\theta)_{ij}\phi_j' \, .
    \end{equation}
    Since $R(\theta)$ is a rotation matrix, it must be orthogonal, i.e.
    \begin{equation}
    \label{orthogonal}
        R^T(\theta) R(\theta) = \mathbb{1} \, ,
        %
        \quad \implies \quad 
        %
        R^T(\theta) = R^{-1}(\theta) \, ,
    \end{equation}
    so we find
    \begin{equation}
        \phi_i = R^{T}(\theta)_{ij}\phi_j' \equiv \phi_j' R(\theta)_{ji} \, .
    \end{equation}
    According to this writing, we obtain
    \begin{equation}
        \partial_\mu \phi_i \partial^\mu \phi_i 
        %
        = \partial_\mu \phi'_j R_{ji}(\theta) \partial^\mu \phi'_k R_{ki}(\theta) 
        %
        = \partial_\mu \phi'_j \partial^\mu \phi'_k R_{ji}(\theta) R_{ki}(\theta) 
        %
        = \partial_\mu \phi_j' \partial^\mu \phi'_k \, ,
    \end{equation}
    where in the last step we used the Eq.~\eqref{orthogonal} to write
    \begin{equation}
        R(\theta)_{ji} R(\theta)_{ki} = R(\theta)_{ji} R^T(\theta)_{ik} = \delta_{ik} \, .
    \end{equation} 
    The same applies to the potential of $\La$:
    \begin{equation}
        \phi_i \phi_i = \phi_j' R(\theta)_{ji} \phi_k' R(\theta)_{ki} = \phi_j' \phi_k' R(\theta)_{ji} R^T(\theta)_{ik} = \phi_j' \phi_j'\,.
    \end{equation}
    In conclusion, we have shown that the transformation in Eq.~\eqref{rot} is a symmetry of the Lagrangian in Eq.~\eqref{lag_ex7}.
    
    In order to determine the \emph{Noether current} and the \emph{Noether charge}, we have to expand the rotation matrix to first order in the parameter $\theta$ as
    \begin{equation}
        R(\theta) =
        \begin{pmatrix}
            \cos \theta & - \sin \theta \\
            \sin \theta & \cos \theta 
        \end{pmatrix}
        =
        \begin{pmatrix}
            1 & 0 \\
            0 & 1 
        \end{pmatrix}
        + \theta
        \begin{pmatrix}
            0 & -1 \\
            1 & 0 
        \end{pmatrix}
        + \order{\theta^2} \, .
    \end{equation}
    It follows that
    \begin{equation}
        \begin{cases}
            \phi_1' = \phi_1 - \theta \phi_2 \, ,\\
            \phi_2' = \phi_2 + \theta \phi_1 \, ,
        \end{cases}
        %
        \quad \Longrightarrow \quad
        \begin{cases}
            \delta\phi_1 = \phi_1' - \phi_1 = - \theta \phi_2 \,, \\
            \delta\phi_2 = \phi_2' - \phi_2 = + \theta \phi_1 \,.
        \end{cases}
    \end{equation}
    Therefore, from the definition of the Noether current we find
    \begin{equation}
        j^\mu 
        %
        = \frac{\partial\La}{\partial\left(\partial_\mu \phi_i\right)} \delta \phi_i 
        %
        = \partial^\mu \phi_1 \, \delta \phi_1 + \partial^\mu \phi_2 \, \delta \phi_2 
        %
        = \phi_1 \, \partial^\mu \phi_2 - \phi_2 \, \partial^\mu \phi_1 \,,
    \end{equation}
    while the Noether charge corresponds to
    \begin{equation}
        Q 
        %
        = \int \rmd^3\vx \, j^0(t, \vx) 
        %
        = \int \rmd^3\vx \left[\phi_1(t, \vx) \dot{\phi_2}(t, \vx) - \phi_2(t, \vx) \dot{\phi_1}(t, \vx)\right] \,.
    \end{equation}
\end{sol}