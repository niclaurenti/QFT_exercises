\begin{ex} \label{ex_14} \addcontentsline{mdf}{mdf}{Exercise \ref{ex_14}}
    Show that the spin operator for the electromagnetic field has the form
    \begin{equation}
        s^{i} = \epsilon^{ijk}\int \rmd^3 \vx \, E_{j}A_{k} = \int \frac{\rmd^3 \vp}{(2\pi)^{3}} \sum_{\lambda ,\lambda'=\pm}\Big[\epsilon^*_{j}(\vp,\lambda)\epsilon_{i}(\vp, \lambda') - \epsilon^*_{i}(\vp,\lambda)\epsilon_{j}(\vp, \lambda')\Big]\acon_{\vp,\lambda}\ades_{\vp, \lambda'} \, .
        \label{Eq_Ex14_spin}
    \end{equation}
\end{ex}

%%%%%%%%%%%%%%%%%%%%%%%%%%%%%%%%%%%%%%%%%%%%%%%%%%%%%%%%%%%%%%%%%%%%%%%%%%%%%%%%%%%%%%%%%%

\begin{sol}
    Let $\Lambda = \Lambda(\theta)$ be a rotation of an angle $\theta$ around the $\hat{z}$ axis. To prove Eq.~\eqref{Eq_Ex14_spin}, we start by computing the conserved charge of $A_{\mu}(x)$. Remember that the conserved current reads
    \begin{equation}
    \label{jmu}
            j^{\mu} = \frac{\partial \mathcal{L}}{\partial(\partial_{\mu}A_{\nu})} \Delta A_{\nu} \,,
    \end{equation}
    where $\Delta A_\nu$ is defined according to
    \begin{equation}
            A_\nu' = A_\nu + \theta \Delta A_{\nu}\,.
    \end{equation}
    The rotation $\Lambda$ acts on $A_\mu(x)$ as 
    \begin{equation}
    \begin{split}
        A'^{\mu}(x) = \tensor{\Lambda}{^\mu_\nu} A^{\nu}(x_{0}) \,, 
        %  
        \qquad 
        %
        x = \Lambda x_{0} \,,
        \end{split}
    \end{equation}
    where
    \begin{equation}
        \tensor{\Lambda}{^\mu_\nu} = \begin{pmatrix}
            1 & 0 & 0 & 0 \\
            0 & 1 & -\theta & 0 \\
            0 & \theta & 1 & 0 \\
            0 & 0 & 0 & 0 \\
        \end{pmatrix} \,,
    \end{equation}
    so we can compute $\Delta A_{\mu}$ as follows
    \begin{equation}
        A'^{\mu}(x) = \tensor{\Lambda}{^\mu_\nu} A^{\nu}(x_{0}) = \Bigl(\id - \frac{i}{2}\tensor{\omega}{_\rho_\sigma}J^{\rho \sigma}\tensor{\Bigr)}{^\mu_\nu} A^{\nu}(\Lambda^{-1} x) \,,
        \label{Eq_Ex14_step1}
    \end{equation}
    with
    \begin{equation}
        \omega_{\rho \sigma} = \begin{pmatrix}
            0 & 0 & 0 & 0 \\
            0 & 0 & \theta & 0 \\
            0 & -\theta & 0 & 0 \\
            0 & 0 & 0 & 0 \\
        \end{pmatrix}\,.
    \end{equation}
    According to the relation $J^i = \frac{1}{2} \epsilon^{ijk} J^{jk}$ and using the antisymmetry of $J^{\rho \sigma}$, Eq.~\eqref{Eq_Ex14_step1} reduces to\footnote{Notice that $J^{3} = \frac{1}{2}(J^{12}-J^{21}) = J^{12}$.} 
    \begin{equation}
    \label{eq_A'mu}
        A'^{\mu}(x) = \bigl(\id - i \omega_{12}J^{3}\tensor{\bigr)}{^\mu _\nu} A^{\nu}(\Lambda^{-1} x)\,.
    \end{equation}
    Now, said $x^\mu = (t,x,y,z)$ and $x_0^\mu = (t,x_0,y_0,z_0)$, to the first order in $\theta$ the spatial coordinates of $x = \Lambda x_0$ are given by
    \begin{equation}
        \begin{cases}
            x = x_0 - \theta y_0 \, , \\
            y = \theta x_0 + y_0 \, , \\
            z = z_0 \, ,
        \end{cases}
    \end{equation}
    so those of $x_0 = \Lambda^{-1} x$ are obtained by changing $\theta$ in $-\theta$, that is
    \begin{equation}
        A^{\nu}(\Lambda^{-1} x) = A^{\nu}(t, x+\theta y, y-\theta x, z) = A^{\nu}(x) + \frac{\partial A^{\nu}}{\partial x}\theta y - \frac{\partial A^{\nu}}{\partial y}\theta x \,.
    \end{equation}
    Plugging this expression into Eq.~\eqref{eq_A'mu}, we get (up to $\mathcal{O}(\theta^{2})$)
    \begin{align}
            A'^{\mu}(x) - A^{\mu}(x) &= (\id - i\theta J^{3}\tensor{)}{^\mu _\nu}\biggl(A^{\nu}(x) + \frac{\partial A^{\nu}}{\partial x}\theta y - \frac{\partial A^{\nu}}{\partial y}\theta x\biggr) - A^{\mu}(x) \notag\\
            &= -i\theta \tensor{(J^3)}{^\mu _\nu} A^{\nu}(x) + \theta(y\partial_{x} - x\partial_{y})A^{\mu}(x) \, .\label{Amu_variation}
    \end{align}
    The first part of Eq.~\eqref{Amu_variation} is a variation related to the internal transformations, while the second one is related to the space-time transformations.
    Since $\Lambda$ is a rotation around the $\hat{z}$ axis, it follows that the spin is the generator of the internal transformations and the orbital angular momentum of the external ones.
    However in this exercise we are interested in the spin, so from now on we only focus on the first part of Eq.~\eqref{Amu_variation}, i.e.
    \begin{equation}
    \label{Amu_variation_spin}
    \Delta A^{\mu} = -i \tensor{(J^3)}{^\mu _\nu} A^{\nu}(x)\,.
    \end{equation}
    Therefore, we plug Eq.~\eqref{Amu_variation_spin} into Eq.~\eqref{jmu}, we use
    \begin{equation}
        \frac{\partial \mathcal{L}}{\partial(\partial_{\mu}A_{\nu})} = -F^{\mu \nu} 
    \end{equation}
    and then set $\mu = 0$ (we are interested in $j^{0}$), which gives $-F^{0\nu} = E^{\nu}$ (with $E^{0} = 0$ because $F^{00} = 0$).
    Putting everything together, we find
    \begin{equation}
        j^{0} = E^{i}\Delta A_{i} = E^{i}(-i)\tensor{(J^3)}{_i_j} A^{j}(x)\, , 
    \end{equation}
    where now $i,j$ run only on spatial components $1,2,3$.  Writing explicitly $J^{3}$ as
    \begin{equation}
        \tensor{(J^3)}{_i_j} = \tensor{(J^{12})}{_i_j} = i(\tensor{\eta}{^1_i}\tensor{\eta}{^2_j} - \tensor{\eta}{^2_j}\tensor{\eta}{^1_i} )\, ,
    \end{equation}
    we get
    \begin{equation}
        j^{0} = E^{1}A^{2} - E^{2}A^{1} = \epsilon^{3ij}E_{i}A_{j}\,,
    \end{equation}
    which means that the conserved charge corresponds to
    \begin{equation}
            Q^{0} \define s^{3} = \int \rmd^3\vx \;\epsilon^{3jk}E_{j}A_{k}\,.
    \end{equation}
    Generalizing this procedure to a generic rotation, one can easily show that 
    \begin{equation}
    \label{eq_si}
        s^{i} = \int \rmd^3\vx \;\epsilon^{ijk}E_{j}A_{k}\, .
    \end{equation}
    Now we only need to substitute the explicit expressions of $E_{j}$ and $A_{k}$ in the last equation. Remember the definitions
    \begin{equation}
    \label{eq_Ak}
        \vA = \int \frac{\rmd^3\vp}{(2\pi)^{3}}\frac{1}{\sqrt{2\omega_{p}}}\sum_{\lambda=\pm}\Big[\veps(p, \lambda)\ades_{\vp, \lambda}e^{-ipx} + \vepsstar(p, \lambda)\acon_{\vp, \lambda}e^{ipx} \Big] \,,
        %
        \qquad
        %
        \vA_{0} = 0 \,,
        %
        \qquad
        %
        \omega_{p} = |\vp|\, ,
    \end{equation}
    and also the relations 
    \begin{equation}
        \vgrad \cdot \vA = 0 \,,
        %
        \qquad
        %
        [\ades_{\vp,\lambda}, \acon_{\vq, \lambda'}] = (2\pi)^{3}\delta^{(3)}(\vp-\vq)\delta_{\lambda \lambda'}\, .
    \end{equation}
    Then, according to the identity $E_{j} = F_{0j} = \partial_{0}A_{j} - \partial_{j}A_{0} = \partial_{0}A_{j}$, we can write $E_j$ as
    \begin{equation}
    \label{eq_Ej}
        \begin{split}
            E_{j} &= \partial_{0}A_{j} = \frac{\partial}{\partial t} \int \frac{\rmd^3\vp}{(2\pi)^{3}}\frac{1}{\sqrt{2\omega_{\vp}}}\sum_{\lambda=\pm}\Big[\epsilon_{j}(\vp, \lambda)\ades_{\vp, \lambda}e^{-ipx} + \epsilon^*_{j}(\vp, \lambda)\acon_{\vp, \lambda}e^{ipx} \Big] \\
            &= i\int \frac{\rmd^3\vp}{(2\pi)^{3}}\sqrt{\frac{\omega_{\vp}}{2}}\sum_{\lambda=\pm}\Big[-\epsilon_{j}(\vp, \lambda)\ades_{\vp, \lambda}e^{-ipx} + \epsilon^*_{j}(\vp, \lambda)\acon_{\vp, \lambda}e^{ipx} \Big]\,.
        \end{split}
    \end{equation}
    At this point we plug both Eqs.~\eqref{eq_Ak} and \eqref{eq_Ej} into Eq.~\eqref{eq_si}, getting
    \begin{equation}
    \label{eq_semifinalsk}
    \begin{split}
            s^{k} = &\; \epsilon^{kji} \int \rmd^3\vx i \int \frac{\rmd^3\vp}{(2\pi)^{3}}\sqrt{\frac{\omega_{\vp}}{2}}\sum_{\lambda=\pm}\Big[-\epsilon_{j}(\vp, \lambda)\ades_{\vp, \lambda}e^{-ipx} + \epsilon^*_{j}(\vp, \lambda)\acon_{\vp, \lambda}e^{ipx} \Big] \\
            %
            &\times \int \frac{\rmd^3\vp'}{(2\pi)^{3}}\frac{1}{\sqrt{2\omega_{\vp'}}}\sum_{\lambda'=\pm}\Big[\epsilon_{i}(\vp', \lambda')\ades_{\vp', \lambda'}e^{-ip'x'} + \epsilon^*_{i}(\vp', \lambda')\acon_{\vp', \lambda'}e^{ip'x'} \Big] \\
            %
            = &\;  i \epsilon^{ijk} \int \frac{\rmd^3\vp}{(2\pi)^{3}} \frac{1}{2}\sum_{\lambda, \lambda'=\pm}\Big[\epsilon^*_{j}(\vp,\lambda)\epsilon_{i}(\vp, \lambda')\acon_{\vp,\lambda}\ades_{\vp, \lambda'} - \epsilon_{j}(\vp,\lambda)\epsilon^*_{i}(\vp, \lambda')\ades_{\vp, \lambda}\acon_{\vp, \lambda'} \\
            %
            & + \epsilon^*_{j}(\vp,\lambda)\epsilon^*_{i}(-\vp, \lambda')\acon_{\vp,\lambda}\acon_{-\vp, \lambda'}e^{2ip^{0}t} - \epsilon_{j}(\vp,\lambda)\epsilon_{i}(-\vp, \lambda')\ades_{\vp,\lambda}\ades_{-\vp, \lambda'}e^{-2ip^{0}t}\Big]\, .
    \end{split}
    \end{equation}
    The last two terms of the last line vanish. In fact, taking for instance the first of the two, we can rewrite it as
    \begin{equation}
    \begin{split}
        & \epsilon^{kji} \epsilon_{j}(\vp,\lambda)\epsilon^*_{i}(-\vp, \lambda')\acon_{\vp, \lambda}\acon_{-\vp \lambda'}  \\
        = &\; \frac{1}{2}\epsilon^{kji}\left(\epsilon^*_{j}(\vp,\lambda)\epsilon^*_{i}(-\vp, \lambda')\acon_{\vp,\lambda}\acon_{-\vp,\lambda'}
        - \epsilon^*_{i}(\vp,\lambda')\epsilon^*_{j}(-\vp, \lambda)\acon_{\vp, \lambda'}\acon_{-\vp, \lambda}\right)\,,
    \end{split}
    \end{equation}
    which is an odd function under $\vp \mapsto - \vp$ and thus gives zero under integration. \par 
    We are then left with the first two terms of Eq.~\eqref{eq_semifinalsk}. Since they are both multiplied by $\epsilon^{kji}$, it is convenient to antisymmetrize them,
    \begin{equation}
    \label{almost_final}
    \begin{split}
        s^{k} = i\epsilon^{kji} \int \frac{\rmd^3\vp}{(2\pi)^{3}}\frac{1}{2}\sum_{\lambda, \lambda'=\pm}\Big[&\left(\epsilon^*_{j}(\vp,\lambda)\epsilon_{i}(p, \lambda') - \epsilon^*_{i}(\vp,\lambda)\epsilon_{j}(\vp, \lambda')\right)\acon_{\vp, \lambda}\ades_{\vp, \lambda'} \\
        &- \left(\epsilon_{j}(\vp,\lambda)\epsilon^*_{i}(\vp, \lambda') - \epsilon_{i}(\vp,\lambda)\epsilon^*_{j}(\vp, \lambda')\right)\ades_{\vp,\lambda}\acon_{\vp, \lambda'}\Big]\, .
    \end{split}
    \end{equation}
    Now it is enough to relabel the summed indices in the last term of Eq.~\eqref{almost_final} as $\lambda \mapsto \lambda'$ and $\lambda' \mapsto \lambda$, and swap the order of the operators $\ades$ and $\acon$ (these operators must be taken inside normal ordering) to get 
    \begin{equation}
        s^{k} = i \epsilon^{kji} \int \frac{\rmd^3\vp}{(2\pi)^{3}}\sum_{\lambda ,\lambda'=\pm}\left[\epsilon^*_{j}(\vp,\lambda)\epsilon_{i}(\vp, \lambda') - \epsilon^*_{i}(\vp,\lambda)\epsilon_{j}(\vp, \lambda')\right]\acon_{\vp,\lambda}\ades_{\vp, \lambda'} \,,
    \end{equation}
    which is exactly Eq.~\eqref{Eq_Ex14_spin}.
\end{sol}